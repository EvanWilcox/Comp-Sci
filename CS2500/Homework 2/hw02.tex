\documentclass[a4paper]{article}

\usepackage[english]{babel}
\usepackage[utf8]{inputenc}
\usepackage{graphicx}
\usepackage{enumitem}
\usepackage{blindtext}
\usepackage{amsmath}
\graphicspath{ {./images/} }
    
\title{CS2500 Homework 2}
\author{Evan Wilcox}
\setlength\parindent{0pt}
    
\date{Due Feburary 7, 2019}
    
\begin{document}
    \maketitle

    \begin{enumerate}

        \item 
        \textbf{2.2-2}
        \begin{verbatim}
selectionSort(A)
1   for i = 1 to A.length-1
2       min = i
3    
4       for j = i+1 to A.length
5           if A[j] < A[min]
6               min = j
7   
8       swap A[i] with A[min]
9
10  return A\end{verbatim}
        
        The subarray $A[1...i-1]$ consists of the smallest elements in sorted order. 
        After the first $n-1$ elements, the subarray A[1...n-1] contains the smallest 
        $n-1$ elements so the $n$th element is the largest element.
        The running time of the algorithm is $\Theta(n^{2})$ for all cases.\\
        

        \textbf{2.2-3}\\
        Average case would be $\Theta(n)$ because the average search time is 
        $\frac{n}{2}$. Worst case would be $\Theta(n)$ because the worst case 
        search time is $n$, or when $n$ is not found.\\
        
        \item
        \textbf{2.3-4}\\
        \[ T(n)=\begin{cases} 
            \Theta(1) & \text{if } n \leq 1, \\
            T(n-1) + D(n) + C(n) & \text{otherwise}. \\
         \end{cases}
        \]\\

         Where $D(n)$ is the time taken to divide the problem and $C(n)$ is the time 
         taken to combine the sub problems.

        \newpage
        \textbf{2.3-5}
         \begin{verbatim}
binarySearch(v, A):
1   if A.length == 0:
2       return -1
3
4   l = 1
5   m = A.length/2
6   h = A.length
7   while(l < m)
8       if v < A[m]
9           h = m
10      else
11          l = m
12          
13      m = (h+l)/2
14
15  if A[m] != v
16      return -1
17
18  return m\end{verbatim}
        The algorithm splits the range in half based on the comparison of the 
        middle element to $v$. The recurrence for this is 
        $T(n) = T(n/2) + \Theta(1)$, whose solution is $T(n) = \Theta(\text{lg}n)$.\\

        \item
        \textbf{2-2}
        \begin{enumerate}[label=d)]
        
            \item Bubblesort's worst-case running time is $\Theta(n^{2})$ which is 
            the same as insertion sort.
        
        \end{enumerate}
    
        \item
        \textbf{3-1}
        \begin{enumerate}[label=\alph*)]
            \item $0 \leq p(n) \leq n^{k}$ for all $n \geq n_0$


            \item


            \item


            \item


            \item

        \end{enumerate}

        \item
        \textbf{3-2}\\
        \begin{tabular}{ ccc|c|c|c|c|c|}
           & $A$ & $B$ & $O$ & $o$ & $\Omega$ & $\omega$ & $\Theta$\\ \hline
        a. & lg$^{k}n$ & $n^{\in}$ &  &  &  &  & \\ \hline
        b. & $n^{k}$ & $c^{n}$ &  &  &  &  & \\ \hline
        d. & $2^{n}$ & $2^{n/2}$ &  &  &  &  & \\ \hline
        e. & $n^{\text{lg}c}$ & $c^{\text{lg}n}$ &  &  &  &  & \\ \hline
        f. & lg$(n!)$ & lg$(n^{n})$ &  &  &  &  & \\ \hline
        \end{tabular}

        \newpage

        \item
        \begin{enumerate}[label=\alph*)]
            
            \item $5n^{2}-6n=\Theta(n^{2})$ \\
            There exist positive constants $c_1, c_2$, and $n_0$ such that 
            $$0 \leq c_1n^{2} \leq 5n^{2}-6n \leq c_2n^{2} \text{ for all } n \geq n_0$$
            Simplified,
            $$0 \leq c_1 \leq 5-\frac{6}{n} \leq c_2$$
            With constants $c_1 = 2, c_2 = 8, n = 12,$
            $$0 \leq 2 \leq 5-\frac{6}{12} \leq 8$$
            $$0 \leq 2 \leq 4\frac{1}{2} \leq 8$$

            \item $n^{3}+10^{6}n^{2}=\Theta(n^{3})$ \\
            There exist positive constants $c_1, c_2$, and $n_0$ such that 
            $$0 \leq c_1n^{3} \leq n^{3}+10^{6}n^{2} \leq c_2n^{3} \text{ for all } n \geq n_0$$
            Simplified,
            $$0 \leq c_1 \leq 1+\frac{10^6}{n} \leq c_2$$
            With constants $c_1 = 2, c_2 = 2, n = 10^{6},$
            $$0 \leq 2 \leq 1+\frac{10^{6}}{10^{6}} \leq 2$$
            $$0 \leq 2 \leq 2 \leq 2$$
    
            \item $6(2^{n}) + n^{2} = O(2^{n})$ \\
            There exist positive constants $c$ and $n_0$ such that 
            $$0 \leq 6(2^{n}) + n^{2} \leq c2^{n} \text{ for all } n \geq n_0$$
            Simplified,
            $$0 \leq 6+\frac{n^{2}}{2^{n}} \leq c$$
            With constants $c = 8, n = 4,$
            $$0 \leq 6+\frac{4^{2}}{2^{4}} \leq 8$$
            $$0 \leq 7 \leq 8$$
        
        \end{enumerate}

        \newpage
        \item
        \begin{enumerate}[label=\alph*)]
            
            \item $10n^{2}+9 \neq \Theta(n)$ 
            There exist positive constants $c_1, c_2$, and $n_0$ such that 
            $$0 \leq c_1n \leq 10n^{2}+9 \leq c_2n \text{ for all } n \geq n_0$$
            Simplified,
            $$0 \leq c_1 \leq 10n+\frac{9}{n} \leq c_2$$
            There is no value for constant $c_2$ large enough to always be greater than 
            $10n+\frac{9}{n}$ for all $n \geq n_0$.

            \vspace{1cm}

            \item $n^{2}lg n \neq \Theta (n^{2})$ \\
            There exist positive constants $c_1, c_2$, and $n_0$ such that 
            $$0 \leq c_1n^{2} \leq n^{2}lg n \leq c_2n^{2} \text{ for all } n \geq n_0$$
            Simplified,
            $$0 \leq c_1 \leq lg n \leq c_2$$
            There is no value for constant $c_2$ large enough to always be greater than 
            $lgn$ for all $n \geq n_0$.
        
        \end{enumerate}

        \vspace{1cm}

        \item $\sum_{i=0}^{n}2^{i}$ for $n=31$ equals 2,147,483,648.
        

        \item $\log_2 1024 = 10$
        

        \item $a^{\log_b c} = b^{(\log_b a)(\log_b c)} = (b^{\log_b (c)})^{\log_b (a)} = c^{\log_b a}$
        
    \end{enumerate}


\end{document}