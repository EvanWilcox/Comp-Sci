\documentclass[a4paper]{article}

\usepackage[english]{babel}
\usepackage[utf8]{inputenc}
\usepackage{graphicx}
\usepackage{enumitem}
\usepackage{blindtext}

\graphicspath{ {./images/} }
\setlength\parindent{0pt}
    
\title{CE2210, Sec. 3 \\Homework 5}
\author{Evan Wilcox}
\date{Due April 12, 2019}
    
\begin{document}
    \maketitle

    \begin{enumerate}
        
      \item  The Set and Reset signals shown below are applied to a NOR-based SR 
      latch. Draw the output waveform $Q(t)$. Assume $Q(t)$ is initially 1. \\
      \vspace{3.5cm}
    
      \item  The signals shown below are applied to a clocked NOR-based SR latch. 
      Draw the output waveform. Assume $Q(t)$ is initially 1. \\
      \vspace{3.5cm}
      
      \item The data signal $D(t)$ shown below is applied to the input of a positive 
      edge-triggered DFF. Draw the output $Q(t)$ for the device. \\

      \newpage
      \item Redo the previous problem for the case where $D(t)$ is applied to the input 
      of a negative edge-triggered DFF. \\
      \vspace{3.5cm}
    
      \item Consider the NAND-based SR latch. What are the outputs, $Q$ and $\bar{Q}$, 
      when both $S = 0$ and $R = 0$? Why is this “not used” case? \\

      When $S = 0$ and $R = 0$ the relationship between $Q$ and $\bar{Q}$ would be 
      wrong because they both would be 1. \\

    
      \item An 8-bit shift register has the binary equivalent of the decimal number 
      46 stored in it. What are the base-10 equivalent contents of the register 
      after the following operations have been performed? \\

      $46_{10}$ = 0010 $1110_{2}$

      \begin{enumerate}
        
        \item SHR 1 = 0001 $0111_{2}$
  
      
        \item SHL 1 = 0101 $1100_{2}$
  
      
        \item SHR 2 = 0000 $1011_{2}$
  
      
        \item ROR 2 = 1000 $1011_{2}$

  
      \end{enumerate}

    \end{enumerate}

\end{document}