\documentclass[a4paper]{article}

\usepackage[english]{babel}
\usepackage[utf8]{inputenc}
\usepackage{graphicx}
\usepackage{enumitem}
\usepackage{blindtext}

\graphicspath{ {./images/} }
\setlength\parindent{0pt}
    
\title{CE2210, Sec. B \\Homework 4}
\author{Evan Wilcox}
\date{Due March 22, 2019}
    
\begin{document}
    \maketitle

    \begin{enumerate}
        
        \item
        \begin{enumerate}

            \item
            

            \item \vspace{4cm}$\frac{1}{0.25(0.8) + 0.5(0.8)}$ = 1.66HZ


            \item 
            \vspace{4cm}

        \end{enumerate}
        \vspace{4cm}

        \item t$_{pO, NAND}$ + 2(t$_{pL, NAND}) + $t$_{pO, NOR} + 2$(t$_{pL, NOR})$ + t$_{pO, NOR}$ \\
        0.85 + 2(0.95) + 0.75 + 2(0.9) + 0.85 = 6.15ns

        \newpage
        \item Construct 4:1 MUX using only 2:1 MUXs as the main building blocks.

        \vspace{10cm}
        \item Use an 8:1 MUX to implement the function: $Q=ab+b\bar{c}$ \\
        
        \newpage
        \item 
        \begin{enumerate}
            
            \item 1100 0110 + 0100 1100 \\
            \begin{tabular}{ c r }
                  & $_{1}$$_{1}$ $_{1}$ $\:$$_{1}$\phantom{spa} \\ 
                  & 1100 0110\\  
                + & 0100 1100\\ \hline
                1 & 0101 0010
            \end{tabular} \\

            \item 1101 0000 + 1010 1010 \\
            \begin{tabular}{ c r }
                & 1101 0000\\  
              + & 1010 1010\\ \hline
              1 & 0111 1010
          \end{tabular} \\
        
        \end{enumerate}

        \item
        \begin{enumerate}

            
            \item 15 - 6 = 15 + (-6) \\ 
            $6_{10} = 0110_{2}$ \\
            2's compliment of 0110: 1010 \\
            \begin{tabular}{ c l }
                & $_{1}$$_{1}$ \\
                & 1111 \\  
              + & 1010 \\ \hline
              1 & 1001 \\
            \end{tabular} \\

            $1001_{2} = 9_{10}$ \\

            \item 196 - 114 = 196 + (-114) \\
            $114_{10} = 0111 0010_{2}$ \\
            2's compliment of 0110: 1000 1110 \\
            \begin{tabular}{ c l }
                & $\:\:\:\:\:\:$ $_{1}$ $_{1}$ \\
                & 1100 0100 \\  
              + & 1000 1110 \\ \hline
              1 & 0101 0010 \\
            \end{tabular} \\

            0101 0010$_{2} = 82_{10}$ \\

        \end{enumerate}

        \item 
        \begin{enumerate}
            
            \item 1011 x 1011 \\
            \begin{tabular}{ c r }
                & 1011 \\  
              x & 1011 \\ \hline
                & 1011 \\
                & 1011\phantom{0} \\
                & 0000\phantom{00} \\
                & 1011\phantom{000} \\ \hline
                & 1111001
            \end{tabular} \\

            \item 1110 x 0110 \\
            \begin{tabular}{ c r }
                & 1110 \\  
              x & 0110 \\ \hline
                & 0000 \\ 
                & 1110\phantom{0} \\
                & 1110\phantom{00} \\
                & 0000\phantom{000} \\ \hline
                & 1010100 \\
            \end{tabular} \\
    
        \end{enumerate}

        \newpage
        \item Design a transmission gate network that implements the function 
        $$F=x\bar{y} + \bar{x}y$$

    \end{enumerate}




\end{document}