\documentclass[a4paper]{article}

\usepackage[english]{babel}
\usepackage[utf8]{inputenc}
\usepackage{graphicx}
\graphicspath{ {./images/} }
    
\title{CS2300 Project Phase 2}
\author{Andrew Henningsen \& Evan Wilcox}
\setlength\parindent{0pt}
    
\date{\today}
    
\begin{document}
    \maketitle
    
    \section{Revised Problem Statement}
    The Overwatch League is a professional esports league for the game of Overwatch developed by and for Blizzard Entertainment.
    Twelve different teams compete in the Pacific and Atlantic Divisions to make it into the playoffs to win a Grand Prize of \$1,000,000.
    Keeping track of all the different players and teams is a difficult task, especially when players are traded or dropped or teams get eliminated 
    from competing. Fans often desire a way to compare different players or in the case that they miss a match, they can view a recap on the go.
    \section{Revised conceptual database design}
    Our database consists of storing information of various different entities: \\

    The "Person" Entity consists of a unique identifying id, a name, and a handle. \\

    There are three different types of "Person", One of which is a "Player". This entity contains a role, a number, a location that they are from,
    and a link to a picture. In addition, a Player can play on many different teams, either the standard twelve teams, or teams for a specific country.    
    
    The second type is "Personnel", casters/personnel play an integral part in the Overwatch e
    \section{Logical database design}

    Team\\
    \begin{tabular}{ |c|c|c|c| }
        \hline
        \underline{id} & name & division & picture\\
        \hline
    \end{tabular}

    \vspace{0.5cm}

    Player\\
    \begin{tabular}{ |c|c|c|c|c|c|c| }
        \hline
        \underline{id} & handle & name & location & no & role & picture\\
        \hline
    \end{tabular}

    \vspace{0.5cm}

    PlayerTeam\\
    \begin{tabular}{ |c|c| }
        \hline
        \underline{playerid} & \underline{teamid}\\
        \hline
    \end{tabular}

    \vspace{0.5cm}

    Match\\
    \begin{tabular}{ |c|c|c|c|c|c| }
        \hline
        \underline{id} & time & score & team1 & team2 & winner\\
        \hline
    \end{tabular}

    \vspace{0.5cm}

    Map Instance\\
    \begin{tabular}{ |c|c|c|c|c| }
        \hline
        \underline{matchid} & \underline{number} & name & time & score\\
        \hline
    \end{tabular}

    \section{Application program design}

    \section{User interface design}
\end{document}